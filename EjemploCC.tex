% Options for packages loaded elsewhere
\PassOptionsToPackage{unicode}{hyperref}
\PassOptionsToPackage{hyphens}{url}
%
\documentclass[
]{article}
\usepackage{amsmath,amssymb}
\usepackage{lmodern}
\usepackage{ifxetex,ifluatex}
\ifnum 0\ifxetex 1\fi\ifluatex 1\fi=0 % if pdftex
  \usepackage[T1]{fontenc}
  \usepackage[utf8]{inputenc}
  \usepackage{textcomp} % provide euro and other symbols
\else % if luatex or xetex
  \usepackage{unicode-math}
  \defaultfontfeatures{Scale=MatchLowercase}
  \defaultfontfeatures[\rmfamily]{Ligatures=TeX,Scale=1}
\fi
% Use upquote if available, for straight quotes in verbatim environments
\IfFileExists{upquote.sty}{\usepackage{upquote}}{}
\IfFileExists{microtype.sty}{% use microtype if available
  \usepackage[]{microtype}
  \UseMicrotypeSet[protrusion]{basicmath} % disable protrusion for tt fonts
}{}
\makeatletter
\@ifundefined{KOMAClassName}{% if non-KOMA class
  \IfFileExists{parskip.sty}{%
    \usepackage{parskip}
  }{% else
    \setlength{\parindent}{0pt}
    \setlength{\parskip}{6pt plus 2pt minus 1pt}}
}{% if KOMA class
  \KOMAoptions{parskip=half}}
\makeatother
\usepackage{xcolor}
\IfFileExists{xurl.sty}{\usepackage{xurl}}{} % add URL line breaks if available
\IfFileExists{bookmark.sty}{\usepackage{bookmark}}{\usepackage{hyperref}}
\hypersetup{
  pdftitle={Ejemplo Carlos De Castilla},
  pdfauthor={Carlos De Castilla},
  hidelinks,
  pdfcreator={LaTeX via pandoc}}
\urlstyle{same} % disable monospaced font for URLs
\usepackage[margin=1in]{geometry}
\usepackage{color}
\usepackage{fancyvrb}
\newcommand{\VerbBar}{|}
\newcommand{\VERB}{\Verb[commandchars=\\\{\}]}
\DefineVerbatimEnvironment{Highlighting}{Verbatim}{commandchars=\\\{\}}
% Add ',fontsize=\small' for more characters per line
\usepackage{framed}
\definecolor{shadecolor}{RGB}{248,248,248}
\newenvironment{Shaded}{\begin{snugshade}}{\end{snugshade}}
\newcommand{\AlertTok}[1]{\textcolor[rgb]{0.94,0.16,0.16}{#1}}
\newcommand{\AnnotationTok}[1]{\textcolor[rgb]{0.56,0.35,0.01}{\textbf{\textit{#1}}}}
\newcommand{\AttributeTok}[1]{\textcolor[rgb]{0.77,0.63,0.00}{#1}}
\newcommand{\BaseNTok}[1]{\textcolor[rgb]{0.00,0.00,0.81}{#1}}
\newcommand{\BuiltInTok}[1]{#1}
\newcommand{\CharTok}[1]{\textcolor[rgb]{0.31,0.60,0.02}{#1}}
\newcommand{\CommentTok}[1]{\textcolor[rgb]{0.56,0.35,0.01}{\textit{#1}}}
\newcommand{\CommentVarTok}[1]{\textcolor[rgb]{0.56,0.35,0.01}{\textbf{\textit{#1}}}}
\newcommand{\ConstantTok}[1]{\textcolor[rgb]{0.00,0.00,0.00}{#1}}
\newcommand{\ControlFlowTok}[1]{\textcolor[rgb]{0.13,0.29,0.53}{\textbf{#1}}}
\newcommand{\DataTypeTok}[1]{\textcolor[rgb]{0.13,0.29,0.53}{#1}}
\newcommand{\DecValTok}[1]{\textcolor[rgb]{0.00,0.00,0.81}{#1}}
\newcommand{\DocumentationTok}[1]{\textcolor[rgb]{0.56,0.35,0.01}{\textbf{\textit{#1}}}}
\newcommand{\ErrorTok}[1]{\textcolor[rgb]{0.64,0.00,0.00}{\textbf{#1}}}
\newcommand{\ExtensionTok}[1]{#1}
\newcommand{\FloatTok}[1]{\textcolor[rgb]{0.00,0.00,0.81}{#1}}
\newcommand{\FunctionTok}[1]{\textcolor[rgb]{0.00,0.00,0.00}{#1}}
\newcommand{\ImportTok}[1]{#1}
\newcommand{\InformationTok}[1]{\textcolor[rgb]{0.56,0.35,0.01}{\textbf{\textit{#1}}}}
\newcommand{\KeywordTok}[1]{\textcolor[rgb]{0.13,0.29,0.53}{\textbf{#1}}}
\newcommand{\NormalTok}[1]{#1}
\newcommand{\OperatorTok}[1]{\textcolor[rgb]{0.81,0.36,0.00}{\textbf{#1}}}
\newcommand{\OtherTok}[1]{\textcolor[rgb]{0.56,0.35,0.01}{#1}}
\newcommand{\PreprocessorTok}[1]{\textcolor[rgb]{0.56,0.35,0.01}{\textit{#1}}}
\newcommand{\RegionMarkerTok}[1]{#1}
\newcommand{\SpecialCharTok}[1]{\textcolor[rgb]{0.00,0.00,0.00}{#1}}
\newcommand{\SpecialStringTok}[1]{\textcolor[rgb]{0.31,0.60,0.02}{#1}}
\newcommand{\StringTok}[1]{\textcolor[rgb]{0.31,0.60,0.02}{#1}}
\newcommand{\VariableTok}[1]{\textcolor[rgb]{0.00,0.00,0.00}{#1}}
\newcommand{\VerbatimStringTok}[1]{\textcolor[rgb]{0.31,0.60,0.02}{#1}}
\newcommand{\WarningTok}[1]{\textcolor[rgb]{0.56,0.35,0.01}{\textbf{\textit{#1}}}}
\usepackage{graphicx}
\makeatletter
\def\maxwidth{\ifdim\Gin@nat@width>\linewidth\linewidth\else\Gin@nat@width\fi}
\def\maxheight{\ifdim\Gin@nat@height>\textheight\textheight\else\Gin@nat@height\fi}
\makeatother
% Scale images if necessary, so that they will not overflow the page
% margins by default, and it is still possible to overwrite the defaults
% using explicit options in \includegraphics[width, height, ...]{}
\setkeys{Gin}{width=\maxwidth,height=\maxheight,keepaspectratio}
% Set default figure placement to htbp
\makeatletter
\def\fps@figure{htbp}
\makeatother
\setlength{\emergencystretch}{3em} % prevent overfull lines
\providecommand{\tightlist}{%
  \setlength{\itemsep}{0pt}\setlength{\parskip}{0pt}}
\setcounter{secnumdepth}{-\maxdimen} % remove section numbering
\ifluatex
  \usepackage{selnolig}  % disable illegal ligatures
\fi

\title{Ejemplo Carlos De Castilla}
\author{Carlos De Castilla}
\date{}

\begin{document}
\maketitle

\hypertarget{ejercicio-carlos-de-castilla-parrilla}{%
\section{Ejercicio Carlos De Castilla
Parrilla}\label{ejercicio-carlos-de-castilla-parrilla}}

\hypertarget{enunciado-del-problema}{%
\subsection{Enunciado del problema}\label{enunciado-del-problema}}

Unos amigos quieren pasar la noche en el casino y jugarse parte de sus
ahorros en alguno de los distintos juegos que se ofrecen en las mesas
del salón quieren decidir en cuál es más rentable jugar. Los distintos
juegos son: El Bingo, El Blackjack, El Póker y La Ruleta.

\hypertarget{planteamiento}{%
\subsection{Planteamiento:}\label{planteamiento}}

\begin{itemize}
\item
  Un único decisor.
\item
  Modelo de beneficios (FAVORABLE).
\item
  Alternativas:

  \begin{itemize}
  \tightlist
  \item
    d1 = ``Bingo''
  \item
    d2 = ``Blackjack''
  \item
    d3 = ``Póker''
  \item
    d4 = ``Ruleta''
  \end{itemize}
\item
  Estados de la Naturaleza:

  \begin{itemize}
  \tightlist
  \item
    e1 = ``Muy mala suerte''
  \item
    e2 = ``Mala suerte''
  \item
    e3 = ``Suerte neutra''
  \item
    e4 = ``Buena suerte''
  \item
    e5 = ``Muy buena suerte''
  \end{itemize}
\item
  A continuación represento la tabla de valores (ganancias en €): Juego:
  (Suerte) // Muy mala // Mala // Neutra // Buena // Muy Buena //\\
  Bingo: // -40 // -30 // -10 // 10 // 30 // Blackjack: // -40 // -20 //
  -10 // 20 // 40 // Póker: // -120 // -60 // -30 // 50 // 150 //
  Ruleta: // -140 // -90 // -10 // 60 // 170 //
\end{itemize}

\hypertarget{resoluciuxf3n}{%
\subsection{Resolución}\label{resoluciuxf3n}}

Como bien nos dice el enunciado del trabajo, el problema ejemplo que hay
crear individualmente tiene que ser resuelto, en primer lugar, empleando
todos los métodos de incertidumbre, y despúes resolverlo con la función
creada por el grupo en conjunto.

Para probar que todos los métodos funcionan podríamos usar ciertas
funciones de la librería de funciones ya creada
``\underline{\emph{teoriadecision\_funciones\_incertidumbre.R}}''

\begin{Shaded}
\begin{Highlighting}[]
\NormalTok{datos}\OtherTok{=}\FunctionTok{c}\NormalTok{(}\SpecialCharTok{{-}}\DecValTok{40}\NormalTok{,}\SpecialCharTok{{-}}\DecValTok{30}\NormalTok{,}\SpecialCharTok{{-}}\DecValTok{10}\NormalTok{,}\DecValTok{10}\NormalTok{,}\DecValTok{30}\NormalTok{,}
        \SpecialCharTok{{-}}\DecValTok{40}\NormalTok{,}\SpecialCharTok{{-}}\DecValTok{20}\NormalTok{,}\SpecialCharTok{{-}}\DecValTok{10}\NormalTok{,}\DecValTok{20}\NormalTok{,}\DecValTok{40}\NormalTok{,}
        \SpecialCharTok{{-}}\DecValTok{120}\NormalTok{,}\SpecialCharTok{{-}}\DecValTok{60}\NormalTok{,}\SpecialCharTok{{-}}\DecValTok{30}\NormalTok{,}\DecValTok{50}\NormalTok{,}\DecValTok{150}\NormalTok{,}
        \SpecialCharTok{{-}}\DecValTok{140}\NormalTok{,}\SpecialCharTok{{-}}\DecValTok{90}\NormalTok{,}\SpecialCharTok{{-}}\DecValTok{10}\NormalTok{,}\DecValTok{60}\NormalTok{,}\DecValTok{170}\NormalTok{)}

\NormalTok{matriz}\OtherTok{=}\FunctionTok{crea.tablaX}\NormalTok{(datos,}\DecValTok{4}\NormalTok{,}\DecValTok{5}\NormalTok{)}
\NormalTok{matriz}
\end{Highlighting}
\end{Shaded}

\begin{verbatim}
##      e1  e2  e3 e4  e5
## d1  -40 -30 -10 10  30
## d2  -40 -20 -10 20  40
## d3 -120 -60 -30 50 150
## d4 -140 -90 -10 60 170
\end{verbatim}

Aplicamos ahora \emph{criterio.Todos} de la librería de funciones
anterior, la cual está cargada al inicio de este fichero, supuesto un
modelo favorable y para un cierto valor de alpha (usamos la función
\emph{criterio.Todos} puesto que es la que nos aplica todos los métodos
de incertidumbre en conjunto sin tener que estar tediosamente aplicando
los métodos uno a uno.

\begin{Shaded}
\begin{Highlighting}[]
\NormalTok{solTodos }\OtherTok{=} \FunctionTok{criterio.Todos}\NormalTok{(}\AttributeTok{tablaX =}\NormalTok{ matriz, }\AttributeTok{alfa =} \FloatTok{0.5}\NormalTok{,}\AttributeTok{favorable =} \ConstantTok{TRUE}\NormalTok{)}
\NormalTok{solTodos}
\end{Highlighting}
\end{Shaded}

\begin{verbatim}
##                   e1  e2  e3 e4  e5  Wald Optimista Hurwicz Savage  Laplace
## d1               -40 -30 -10 10  30   -40        30      -5    140       -8
## d2               -40 -20 -10 20  40   -40        40       0    130       -2
## d3              -120 -60 -30 50 150  -120       150      15     80       -2
## d4              -140 -90 -10 60 170  -140       170      15    100       -2
## iAlt.Opt (fav.)   --  --  -- --  -- d1,d2        d4   d3,d4     d3 d2,d3,d4
##                 Punto Ideal
## d1                   149.00
## d2                   136.01
## d3                    94.34
## d4                   122.07
## iAlt.Opt (fav.)          d3
\end{verbatim}

Tenemos por tanto los siguientes resultados:

\begin{itemize}
\tightlist
\item
  Para el criterio Pesimista o de Wald la decisión correcta es Bingo o
  Blackjack, para el criterio optimista la mejor opción es la Ruleta,
  para el criterio de Hurwicz, la mejor alternativa sería jugar al Póker
  o a la Ruleta.
\end{itemize}

-Aplicando el método de Savage obtenemos que la alternativa de mínimo
arrepentimiento sería jugar al Póker. Se jugaría también al Póker si
nuestro método fuese el del Punto Ideal.

-Y por último si aplicáramos el criterio de Laplace (sucesos
equiprobables) obtendríamos un empate entre las alternativas de jugar a
Blackjack, Póker o Ruleta, por lo que se puede deducir que da igual
jugar a cualquiera de esos 3 juegos.

\hypertarget{resoluciuxf3n-utilizando-la-funciuxf3n-creada-por-el-grupo}{%
\subsection{Resolución utilizando la función creada por el
grupo}\label{resoluciuxf3n-utilizando-la-funciuxf3n-creada-por-el-grupo}}

Haremos uso de nuestra función para obtener los distintos intervalos del
alfa (alfa mide nuestra valentía siendo 0 muy pesimista y 1 muy
optimista) en los que cambian las alternativas óptima. siendo el
criterio de incertidumbre el método de \textbf{Hurwicz}. La función es
la siguiente:

\begin{Shaded}
\begin{Highlighting}[]
\CommentTok{\# Función Principal (creada por el grupo)}
\NormalTok{intervalos.alfa}\OtherTok{=} \ControlFlowTok{function}\NormalTok{(tablaX,}\AttributeTok{favorable=}\ConstantTok{TRUE}\NormalTok{) \{}
  
\NormalTok{  alfa}\OtherTok{=}\FunctionTok{seq}\NormalTok{(}\DecValTok{0}\NormalTok{,}\DecValTok{1}\NormalTok{,}\AttributeTok{by=}\FloatTok{0.01}\NormalTok{) }\CommentTok{\# Introducimos un conjunto de alfas que nos servirán}
  \CommentTok{\# para saber cuándo cambia la alternativa óptima. Fijamos un valor de 0.01,}
  \CommentTok{\# el cual indica cada cuánto se quiere que exista alfa.}
\NormalTok{  X }\OtherTok{=}\NormalTok{ tablaX}
  
  \ControlFlowTok{if}\NormalTok{(favorable)\{ }\CommentTok{\#en el caso de que sea favorable}
\NormalTok{    Altmin }\OtherTok{=} \FunctionTok{apply}\NormalTok{(X,}\AttributeTok{MARGIN=}\DecValTok{1}\NormalTok{,min)}
\NormalTok{    Altmax}\OtherTok{=} \FunctionTok{apply}\NormalTok{(X,}\AttributeTok{MARGIN=}\DecValTok{1}\NormalTok{,max)}
    
    \CommentTok{\# Como ya no tenemos un alfa, sino varios, debemos crear un bucle que}
    \CommentTok{\# trabaje con todos los alfa. Como, además, Altmin y Altmax son vecto{-}}
    \CommentTok{\# res, tenemos que crear listas que nos devuelvan, para cada elemento}
    \CommentTok{\# de dichos vectores, cuáles son las alternativas asociadas a los}
    \CommentTok{\# alfas.}
    
\NormalTok{    AltH}\OtherTok{=}\FunctionTok{list}\NormalTok{()}
\NormalTok{    Hurwicz}\OtherTok{=}\FunctionTok{list}\NormalTok{()}
\NormalTok{    Alt\_Hurwicz}\OtherTok{=}\FunctionTok{list}\NormalTok{()}
    
    \CommentTok{\# Creamos el bucle:}
    
    \ControlFlowTok{for}\NormalTok{(i }\ControlFlowTok{in} \DecValTok{1}\SpecialCharTok{:}\FunctionTok{length}\NormalTok{(alfa))\{}
\NormalTok{      AltH[[i]] }\OtherTok{=}\NormalTok{ alfa[i] }\SpecialCharTok{*}\NormalTok{ Altmax }\SpecialCharTok{+}\NormalTok{ (}\DecValTok{1}\SpecialCharTok{{-}}\NormalTok{alfa[i]) }\SpecialCharTok{*}\NormalTok{ Altmin}
\NormalTok{      Hurwicz[[i]] }\OtherTok{=} \FunctionTok{max}\NormalTok{(AltH[[i]])}
\NormalTok{      Alt\_Hurwicz[[i]] }\OtherTok{=} \FunctionTok{which.max.general}\NormalTok{(AltH[[i]])}
\NormalTok{    \}}
    
\NormalTok{    metodo }\OtherTok{=} \StringTok{\textquotesingle{}favorable\textquotesingle{}}
    
\NormalTok{  \} }\ControlFlowTok{else}\NormalTok{ \{ }\CommentTok{\#en caso de que no sea favorable}
\NormalTok{    Altmin }\OtherTok{=} \FunctionTok{apply}\NormalTok{(X,}\AttributeTok{MARGIN=}\DecValTok{1}\NormalTok{,min)}
\NormalTok{    Altmax}\OtherTok{=} \FunctionTok{apply}\NormalTok{(X,}\AttributeTok{MARGIN=}\DecValTok{1}\NormalTok{,max)}
    
\NormalTok{    AltH}\OtherTok{=}\FunctionTok{list}\NormalTok{()}
\NormalTok{    Hurwicz}\OtherTok{=}\FunctionTok{list}\NormalTok{()}
\NormalTok{    Alt\_Hurwicz}\OtherTok{=}\FunctionTok{list}\NormalTok{()}
    
    \ControlFlowTok{for}\NormalTok{(i }\ControlFlowTok{in} \DecValTok{1}\SpecialCharTok{:}\FunctionTok{length}\NormalTok{(alfa))\{}
\NormalTok{      AltH[[i]] }\OtherTok{=}\NormalTok{ (}\DecValTok{1}\SpecialCharTok{{-}}\NormalTok{alfa[i]) }\SpecialCharTok{*}\NormalTok{ Altmax }\SpecialCharTok{+}\NormalTok{ alfa[i] }\SpecialCharTok{*}\NormalTok{ Altmin }
\NormalTok{      Hurwicz[[i]] }\OtherTok{=} \FunctionTok{min}\NormalTok{(AltH[[i]])}
\NormalTok{      Alt\_Hurwicz[[i]] }\OtherTok{=} \FunctionTok{which.min.general}\NormalTok{(AltH[[i]])}
\NormalTok{    \}}
    
\NormalTok{    metodo }\OtherTok{=} \StringTok{\textquotesingle{}desfavorable\textquotesingle{}}
    
\NormalTok{  \}}
  
\NormalTok{  resultados }\OtherTok{=} \FunctionTok{list}\NormalTok{();}
\NormalTok{  resultados}\SpecialCharTok{$}\NormalTok{metodo }\OtherTok{=}\NormalTok{ metodo;}
\NormalTok{  resultados}\SpecialCharTok{$}\NormalTok{ValorAlternativa }\OtherTok{=} \FunctionTok{unlist}\NormalTok{(Hurwicz); }\CommentTok{\# Valores que toma cada}
  \CommentTok{\# alfa en su alternativa óptima.}
\NormalTok{  resultados}\SpecialCharTok{$}\NormalTok{alfa }\OtherTok{=}\NormalTok{ alfa; }\CommentTok{\# Alfas usados.}
\NormalTok{  resultados}\SpecialCharTok{$}\NormalTok{AlternativaOptima }\OtherTok{=} \FunctionTok{unlist}\NormalTok{(Alt\_Hurwicz);}
\NormalTok{  resultados}\SpecialCharTok{$}\NormalTok{Solucion }\OtherTok{=} \FunctionTok{unlist}\NormalTok{(Alt\_Hurwicz);}
  \FunctionTok{names}\NormalTok{(resultados}\SpecialCharTok{$}\NormalTok{Solucion)}\OtherTok{=}\NormalTok{alfa;}
  \CommentTok{\# distinct(as.data.frame(resultados,alfa));}
\NormalTok{  prueba}\OtherTok{=}\FunctionTok{cbind.data.frame}\NormalTok{(alfa,resultados}\SpecialCharTok{$}\NormalTok{Solucion[}\DecValTok{1}\SpecialCharTok{:}\FunctionTok{length}\NormalTok{(alfa)]);}
  \FunctionTok{colnames}\NormalTok{(prueba)}\OtherTok{\textless{}{-}}\FunctionTok{c}\NormalTok{(}\StringTok{"Alfa"}\NormalTok{, }\StringTok{"Solución"}\NormalTok{);}
  \FunctionTok{rownames}\NormalTok{(prueba }\SpecialCharTok{\%\textgreater{}\%} \FunctionTok{distinct}\NormalTok{(Solución));}
\NormalTok{  alternativas}\OtherTok{=}\NormalTok{ prueba }\SpecialCharTok{\%\textgreater{}\%} \FunctionTok{distinct}\NormalTok{(Solución);}
\NormalTok{  resultados}\SpecialCharTok{$}\NormalTok{intervalos }\OtherTok{=}\NormalTok{ alternativas}
  \FunctionTok{return}\NormalTok{(resultados)}
\NormalTok{\}}
\end{Highlighting}
\end{Shaded}

La aplicamos a nuestra matriz de datos:

\begin{Shaded}
\begin{Highlighting}[]
\NormalTok{solHurwicz }\OtherTok{=} \FunctionTok{intervalos.alfa}\NormalTok{(matriz, }\AttributeTok{favorable =}\NormalTok{ T)}
\NormalTok{solHurwicz}
\end{Highlighting}
\end{Shaded}

\begin{verbatim}
## $metodo
## [1] "favorable"
## 
## $ValorAlternativa
##   [1] -40.0 -39.2 -38.4 -37.6 -36.8 -36.0 -35.2 -34.4 -33.6 -32.8 -32.0 -31.2
##  [13] -30.4 -29.6 -28.8 -28.0 -27.2 -26.4 -25.6 -24.8 -24.0 -23.2 -22.4 -21.6
##  [25] -20.8 -20.0 -19.2 -18.4 -17.6 -16.8 -16.0 -15.2 -14.4 -13.6 -12.8 -12.0
##  [37] -11.2 -10.4  -9.6  -8.8  -8.0  -7.2  -6.4  -3.9  -1.2   1.5   4.2   6.9
##  [49]   9.6  12.3  15.0  18.1  21.2  24.3  27.4  30.5  33.6  36.7  39.8  42.9
##  [61]  46.0  49.1  52.2  55.3  58.4  61.5  64.6  67.7  70.8  73.9  77.0  80.1
##  [73]  83.2  86.3  89.4  92.5  95.6  98.7 101.8 104.9 108.0 111.1 114.2 117.3
##  [85] 120.4 123.5 126.6 129.7 132.8 135.9 139.0 142.1 145.2 148.3 151.4 154.5
##  [97] 157.6 160.7 163.8 166.9 170.0
## 
## $alfa
##   [1] 0.00 0.01 0.02 0.03 0.04 0.05 0.06 0.07 0.08 0.09 0.10 0.11 0.12 0.13 0.14
##  [16] 0.15 0.16 0.17 0.18 0.19 0.20 0.21 0.22 0.23 0.24 0.25 0.26 0.27 0.28 0.29
##  [31] 0.30 0.31 0.32 0.33 0.34 0.35 0.36 0.37 0.38 0.39 0.40 0.41 0.42 0.43 0.44
##  [46] 0.45 0.46 0.47 0.48 0.49 0.50 0.51 0.52 0.53 0.54 0.55 0.56 0.57 0.58 0.59
##  [61] 0.60 0.61 0.62 0.63 0.64 0.65 0.66 0.67 0.68 0.69 0.70 0.71 0.72 0.73 0.74
##  [76] 0.75 0.76 0.77 0.78 0.79 0.80 0.81 0.82 0.83 0.84 0.85 0.86 0.87 0.88 0.89
##  [91] 0.90 0.91 0.92 0.93 0.94 0.95 0.96 0.97 0.98 0.99 1.00
## 
## $AlternativaOptima
## d1 d2 d2 d2 d2 d2 d2 d2 d2 d2 d2 d2 d2 d2 d2 d2 d2 d2 d2 d2 d2 d2 d2 d2 d2 d2 
##  1  2  2  2  2  2  2  2  2  2  2  2  2  2  2  2  2  2  2  2  2  2  2  2  2  2 
## d2 d2 d2 d2 d2 d2 d2 d2 d2 d2 d2 d2 d2 d2 d2 d2 d2 d2 d3 d3 d3 d3 d3 d3 d3 d3 
##  2  2  2  2  2  2  2  2  2  2  2  2  2  2  2  2  2  2  3  3  3  3  3  3  3  3 
## d4 d4 d4 d4 d4 d4 d4 d4 d4 d4 d4 d4 d4 d4 d4 d4 d4 d4 d4 d4 d4 d4 d4 d4 d4 d4 
##  4  4  4  4  4  4  4  4  4  4  4  4  4  4  4  4  4  4  4  4  4  4  4  4  4  4 
## d4 d4 d4 d4 d4 d4 d4 d4 d4 d4 d4 d4 d4 d4 d4 d4 d4 d4 d4 d4 d4 d4 d4 d4 d4 
##  4  4  4  4  4  4  4  4  4  4  4  4  4  4  4  4  4  4  4  4  4  4  4  4  4 
## 
## $Solucion
##    0 0.01 0.02 0.03 0.04 0.05 0.06 0.07 0.08 0.09  0.1 0.11 0.12 0.13 0.14 0.15 
##    1    2    2    2    2    2    2    2    2    2    2    2    2    2    2    2 
## 0.16 0.17 0.18 0.19  0.2 0.21 0.22 0.23 0.24 0.25 0.26 0.27 0.28 0.29  0.3 0.31 
##    2    2    2    2    2    2    2    2    2    2    2    2    2    2    2    2 
## 0.32 0.33 0.34 0.35 0.36 0.37 0.38 0.39  0.4 0.41 0.42 0.43 0.44 0.45 0.46 0.47 
##    2    2    2    2    2    2    2    2    2    2    2    2    3    3    3    3 
## 0.48 0.49  0.5 0.51 0.52 0.53 0.54 0.55 0.56 0.57 0.58 0.59  0.6 0.61 0.62 0.63 
##    3    3    3    3    4    4    4    4    4    4    4    4    4    4    4    4 
## 0.64 0.65 0.66 0.67 0.68 0.69  0.7 0.71 0.72 0.73 0.74 0.75 0.76 0.77 0.78 0.79 
##    4    4    4    4    4    4    4    4    4    4    4    4    4    4    4    4 
##  0.8 0.81 0.82 0.83 0.84 0.85 0.86 0.87 0.88 0.89  0.9 0.91 0.92 0.93 0.94 0.95 
##    4    4    4    4    4    4    4    4    4    4    4    4    4    4    4    4 
## 0.96 0.97 0.98 0.99    1 <NA> <NA> 
##    4    4    4    4    4    4    4 
## 
## $intervalos
##      Solución
## 0           1
## 0.01        2
## 0.44        3
## 0.52        4
\end{verbatim}

Obtenemos los siguientes valores de Alfa a partir de los cuales
cambiamos de alternativa.

Si nuestro alfa va de {[}0,0.01) la mejor alternativa es el Bingo. Si
nuestro alfa está entre {[}0.01,0.44) lo mejor es jugar al Blackjack. Si
nuestro alfa va de {[}0.44,0.52) la mejor decisión es jugar al Póker. Y
si nuestro alfa va de {[}0.52,1{]} el mejor plan es jugar a la Ruleta.

Lo podemos comprobar realizando una \underline{representación} gráfica
mediante la función \textbf{\emph{dibuja.criterio.Hurwicz}} de la
librería ``\emph{teoriadecision\_funciones\_incertidumbre}'':

\begin{Shaded}
\begin{Highlighting}[]
\FunctionTok{dibuja.criterio.Hurwicz}\NormalTok{(matriz, }\AttributeTok{favorable =}\NormalTok{ T)}
\end{Highlighting}
\end{Shaded}

\includegraphics{EjemploCC_files/figure-latex/unnamed-chunk-5-1.pdf}

\end{document}
